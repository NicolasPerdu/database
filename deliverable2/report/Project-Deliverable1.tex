\documentclass[a4paper,10pt]{article}

\usepackage[utf8]{inputenc}
\usepackage{graphicx}
\usepackage{listings}
\usepackage[usenames,dvipsnames]{color}

\title{Deliverable 1}
\author{Chervine Majeri, Nicolas Perdu, Yannick Schaeffer}
\date{20.03.2014}

\pdfinfo{%
  /Title    (Deliverable 1)
  /Author   (Chervine Majeri, Nicolas Perdu, Yannick Schaeffer)
  /Creator  ()
  /Producer ()
  /Subject  (Database Systems)
  /Keywords ()
}

\lstset{
basicstyle=\normalsize,
keywordstyle=\ttfamily\color{OliveGreen},
identifierstyle=\ttfamily\color{CadetBlue}\bfseries,
commentstyle=\color{Brown},
stringstyle=\ttfamily,
showstringspaces=true
}

\begin{document}
\maketitle
\section{Entity-Relationship Diagram}
Let's put words on the data, discussing what could be entities, or relationships (not attributes).\\
\begin{itemize}
\item An artist comes from an area.
\item An artist plays in a set of genres.
\item An artist can write tracks.
\item Tracks are recordings, present on the physical/digital port of a release, at some position.
\end{itemize}
\begin{figure}[ht!]
\centering
\includegraphics[width=90mm]{ERD.jpeg}
\caption{Entity-Relationship Diagram}
\label{overflow}
\end{figure}
\newpage
The only decision that needs explaining is the fact that Track is a relation, and not an entity.\\
Here, an analogy with the real world is necessary. A track is usually considered to be a recording. 
Its physical support doesn't affect the property of the object : it is a piece of music.\\
The same way, a port of an album is just that, a port. We have different ports of the same object (the release) on different formats (the medium).\\
So a recording is the x-th track on some port of a release. Hence the aggregations.\\
Note that not all recordings have made it to be released.\\
\section{Key/Participation Constraints}
An artist can only come from one area, however, an area can have spawned many artists.\\
There is a Many-to-one key constraint between Artist and Area.\\
An artist is considered an artist only after he has started producing music. This music will have a genre (even if it is a brand new one).\\
There is a total participation constraint on Artists and the "Plays in" relation.\\
What has been written is always a track, and a track has always been written : total participation constraint between Track and Wrote.\\
A port always has a track (this is a supposition), and a track always belongs to a port : total participation constraint between Track and Port.\\

An artist can still be an artist even though he hasn't released any tracks, so there can't be a constraint there.\\
\section{Tables creation}
\lstset{language=SQL}
\begin{lstlisting}[frame=single, float=h]
CREATE TABLE Area (
  areaID	NUMBER (12),
  name		VARCHAR2 (80),
  type		CHAR (12),
  PRIMARY KEY (areaID)
)
\end{lstlisting}
\begin{lstlisting}[frame=single, float=h]
CREATE TABLE Artist (
  artistID	NUMBER (12),
  name		VARCHAR2 (80),
  type		CHAR (6),
  gender	CHAR (6),
  areaID	NUMBER (12),
  PRIMARY KEY (artistID),
  FOREIGN KEY (areaID)
    REFERENCES Area(areaID)
    ON DELETE SET NULL
)
\end{lstlisting}
\begin{lstlisting}[frame=single, float=h]
CREATE TABLE Plays_in (
  artistID	NUMBER (12),
  genreID	NUMBER (12),
  PRIMARY KEY (artistID, genreID),
  FOREIGN KEY (artistID)
    REFERENCES Artist(artistID)
    ON DELETE CASCADE,
  FOREIGN KEY (genreID)
    REFERENCES Genre(genreID)
    ON DELETE CASCADE
)
\end{lstlisting}
\begin{lstlisting}[frame=single, float=h]
CREATE TABLE Genre (
  genreID	NUMBER (12),
  name		VARCHAR2 (80),
  count		NUMBER (8),
  PRIMARY KEY (genreID)
)
\end{lstlisting}
\begin{lstlisting}[frame=single, float=h]
CREATE TABLE Wrote (
  artistID	NUMBER (12),
  trackID	NUMBER (12),
  PRIMARY KEY (artistID, trackID),
  FOREIGN KEY (artistID)
    REFERENCES Artist(artistID)
    ON DELETE CASCADE,
  FOREIGN KEY (trackID)
    REFERENCES Track(trackID)
    ON DELETE CASCADE
)
\end{lstlisting}
\begin{lstlisting}[frame=single, float=h]
CREATE TABLE Recording (
  recordingID	NUMBER (12),
  name		VARCHAR2 (80),
  length	NUMBER (12),
  PRIMARY KEY (recordingID)
)
\end{lstlisting}
\begin{lstlisting}[frame=single, float=h]
CREATE TABLE Track (
  trackID	NUMBER (12),
  recordingID	NUMBER (12),
  releaseID	NUMBER (12),
  format	CHAR (20),
  position	NUMBER (4),
  PRIMARY KEY (trackID),
  UNIQUE (recordingID, releaseID, format),
  FOREIGN KEY (recordingID),
    REFERENCES Recording(recordingID)
    ON DELETE CASCADE,
  FOREIGN KEY (releaseID)
    REFERENCES Release(releaseID)
    ON DELETE CASCADE,
)
\end{lstlisting}
\begin{lstlisting}[frame=single, float=h]
CREATE TABLE Release (
  releaseID	NUMBER (12),
  name		VARCHAR2 (80),
  PRIMARY KEY (releaseID)
)
\end{lstlisting}
\section{Interface}
The php extension \textit{php_oci8} should be enabled.\\
\\
The interface is done in PHP in two files :\\
index.php - display the result when a query is send with the form.\\
func.php - contain the main function: \textit{page($query)}, compute the result of the query and return an array.\\
\\
The function \textit{page($query)} is simple, we use oci_connect to obtain a connexion.\\
We use \textit{oci_parse} to parse the query given as parameter, and then we execute \\
the query with \textit{oci_execute} and get the result with \textit{oci_fetch_array} \\
\\
If the query is performed on the table Medium, instead of displaying the mediumID, \\
we create a form with a hidden button, so that the user can know more about track with this id, \\
the query perform is \textit{SELECT * FROM Track WHERE mediumId = id}\\
\\
The same thing is done with a query on Release, we display \textit{SELECT * FROM Medium WHERE releaseId = id}\\

\end{document}
